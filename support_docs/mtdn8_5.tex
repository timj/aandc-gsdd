\documentstyle{article} 
\pagestyle{myheadings}

%------------------------------------------------------------------------------
\newcommand{\jachdoccategory}  {Design Note}
\newcommand{\jachdocinitials}  {MTDN}
\newcommand{\jachdocnumber}    {008.5}
\newcommand{\jachdocauthors}   {James Scobbie}
\newcommand{\jachdocdate}      {August 4th, 1994}
\newcommand{\jachdoctitle}     {JCMT Single Dish Data System}
%------------------------------------------------------------------------------

\newcommand{\jachdocname}{\jachdocinitials /\jachdocnumber}
\markright{\jachdocname}
\setlength{\textwidth}{160mm}
\setlength{\textheight}{225mm}
\setlength{\topmargin}{-5mm}
\setlength{\oddsidemargin}{0mm}
\setlength{\evensidemargin}{0mm}
\setlength{\parindent}{0mm}
\setlength{\parskip}{\medskipamount}
\setlength{\unitlength}{1mm}

%------------------------------------------------------------------------------
% Add any \newcommand or \newenvironment commands here
%------------------------------------------------------------------------------

\begin{document}
\thispagestyle{empty}
SCIENCE \& ENGINEERING RESEARCH COUNCIL \hfill \jachdocname\\
JOINT ASTRONOMY CENTRE\\
{\large\bf James Clerk Maxwell Telescope\\}
{\large\bf \jachdoccategory\ \jachdocnumber}
\begin{flushright}
\jachdocauthors\\
\jachdocdate
\end{flushright}
\vspace{-4mm}
\rule{\textwidth}{0.5mm}
\vspace{5mm}
\begin{center}
{\Large\bf \jachdoctitle}
\end{center}
\vspace{5mm}

%------------------------------------------------------------------------------
%  Add this part if you want a table of contents
 \setlength{\parskip}{0mm}  
 \tableofcontents
 \setlength{\parskip}{\medskipamount}
 \markright{\jachdocname}
%------------------------------------------------------------------------------


\section{Introduction}
\subsection{Purpose}

  This document provides a description of the current use of
the General Single Dish Data Format System at the JCMT.


\subsection{Scope}

  The scope includes all data stored in General Single Dish Data Format
at the JCMT.

\subsection{Definitions and abbreviations}
\begin{description}
\item [GSD]      General Single Dish Data handling routines, JCMT only.
\item [GSDD]     General Single Dish Data Format.
\item [GSDF]     General Single Dish JCMT Data definition file.  
\item [JCMT]     James Clerk Maxwell Telescope. 
\item [ADAM]     Astronomical Data Aquisition and Monitor.
\item [NBS]      Noticeboard System from the Starlink collection.
\item [FITS]     Flexible Image Transport System.
\end{description}


\subsection{References}

\begin{enumerate}
  \item {\bf MTDN008} General Single Dish Data Format System, Version 4.  
        {\em Fairclough,Kenderdine,Titterington 23rd Jan 1987}
  \item {\bf NRAO} Single Dish Data Format {\em Brown, Stobie 21st Oct 1986}
  \item {\bf NRAO} Single Fits Tape {\em Stobie, Morgan 21st Oct 1986}
\end{enumerate}




\subsection{Overview}

  The General Single Dish Data Format System specifies the physical
format, structure and type of information that comprise observations as 
agreed by NRAO, IRAM and UK/NL MT teams at a meeting of these parties
back in 1986.

  This format system has since failed to be adopted by any of the 
organisations that participated in its creation. The Format itself has
not been developed further, however at JCMT a similar system based upon 
the General Single Dish Data Format System has evolved. 

  Some of the reasoning behind the implementation route taken was based 
upon the dominance of a single operating system (VMS), the lack of a 
developed or applicable FITS standard (with BINTABLES) and the performance 
limitations of the hardware of that period.

  With time the system at JCMT has actively evolved without plan, 
control or documentation. This document attempts to define what is currently 
done at the JCMT in terms of data storage content and is to provide a
basis on which the future of the system may be decided.




\section{GSD Architecture}

  The GSD Architecture described in MTDN008 Version 4 (Reference 1)   
remains the same. Each file follows the format described starting with
the single file descriptor, an item descriptor for each datum, and then
the data. 


\section{JCMT Storage Data Definitions}

   The JCMT system of Storing data in GSD files utilizes a set of user
supplied files containing data definitions. These files (GSDF files)
define what data is to be stored by the Observing system STORAGE task.

   Each data definition includes,

\begin{description}
  \item [COMMENT]    A character string describing the item.
  \item [FITS NAME]  A JCMT specific FITS keyword.
  \item [GSDD NAME]  The General Single Dish Data Format item identifier.
  \item [CLASS]      A mask used to identify items stored together.
  \item [TYPE]       The data type (Char,Integer,Real,Double,Logical) 
  \item [LOCATION]   The Adam NBS location of the data item.
  \item [DIMENSION]  Dimensions of array, zero being a scalar.
  \item [DIMENSIONS] The GSDD NAME of each item that will contain the value 
                     of a dimension at run time.
\end{description}

   There exist 17 top level GSDF files which are composed of 94 component 
files defining a total of nearly 600 seperate items. Many of these are
historical.

   For this document and for the current definition of the data items to
be stored only the following GSDF files will be considered as "in use".

\begin{itemize}
  \item STORAGE\_CBE\_FULL.GSDF
  \item STORAGE\_AOSC\_FULL.GSDF
  \item STORAGE\_UKT14\_FULL.GSDF
  \item STORAGE\_IFD\_FULL.GSDF
  \item STORAGE\_DAS\_FULL.GSDF
  \item STORAGE\_DAS\_TP\_FULL.GSDF
  \item STORAGE\_DAS\_CROSS\_CORR\_FULL.GSDF
\end{itemize}



\newpage
\section{The Class System}

  The Class system is described in MTDN008 Version 4 (Reference 1) 
where three Systems of classes are explained, the Simple System, 
System 1 and System 2. A class is a grouping of similar data items
under a naming convention.


  Each of the three Systems has a different number of classes where
the classes are not the same between Systems. For continuity, for this
version, the same Class headings as those of System 1 and System 2 have
been used. The two additional Classes are Class 55 which is used for
data related to dual dish interferometry and the No Class which 
contains data items that do not currently belong to any other class.


\small
\subsection{Class 1: Identity Parameters}


\begin{tabular}{||l|l|l|l||} \hline
Item                      & Mnemonic & Format & Units \\ \hline
Telescope name            & C1TEL    & C*16   &       \\      
Identifies the observing program & C1PID    & C*16   & ~     \\
Name of the primary observer     & C1OBS    & C*16   & ~     \\
Name of the support scientist    & C1ONA1   & C*16   & ~     \\
Name of the telescope operator   & C1ONA2   & C*16   & ~     \\
Source name part 1               & C1SNA1   & C*16   & ~     \\
Source name part 2               & C1SNA2   & C*16   & ~     \\
Character code of commanded centre  & C1HGT    & R*8    & KM    \\
Geographical longitude of telescope (W +ve) & C1LONG   & R*8    & DEG   \\
Geodetic latitude of telescope (N +ve)      & C1LAT    & R*8    & DEG   \\
Observation number        & C1SNO    & R*8    &  ~    \\
Name of the frontend      & C1RCV    & C*16   &  ~    \\
Type of frontend          & C1FTYP   & C*16   &  ~    \\
Name of the backend       & C1BKE    & C*16   &  ~    \\
Type of backend           & C1BTYP   & C*16   &  ~    \\ \hline
\end{tabular}


 C1HGT is Character code of commanded centre or source coordinate system. \\
 C1SNO is listed as D ( Real*8 ) in the GSDF files but as an Integer in control task.





\subsection{Class 2: Time Parameters}


\begin{tabular}{||l|l|l|l||} \hline
Item                      & Mnemonic & Format & Units \\ \hline
Secondary mirror x displacement from nominal at observation start  & C2FV     & R*4    &  MM   \\
Secondary mirror y displacement from nominal at observation start  & C2FL     & R*4    &  MM   \\
Secondary mirror z displacement from nominal at observation start  & C2FR     & R*4    &  MM   \\ \hline
\end{tabular}
 \\




\subsection{Class 3: Position Parameters}


\begin{tabular}{||l|l|l|l||} \hline
Item                      & Mnemonic & Format & Units \\ \hline
UT1 date of observation   & C3DAT    & R*8    &  YYYY.MMDD \\      
UT1 hour of observation   & C3UT     & R*8    & HOUR  \\      
UT1-UTC correction interpolated from time service telex & C3UT1C   & R*8    & DAY   \\      
Local sidereal time at start of the observation & C3LST    & R*8    & HOUR  \\      
Calibration observation?    & C3CAL    & L*1    & ~     \\      
Centre moves between scans? & C3CEN    & L*1    & ~     \\      
Data taken on the fly or in discrete mode? & C3FLY    & L*1    & ~     \\      
Focus observation?        & C3FOCUS  & L*1    &  ~    \\      
Map observation?          & C3MAP    & L*1    &  ~    \\      
Number of dimension in the map table & C3NPP    & I*4    &  ~    \\      
Number of map points      & C3NMAP   & I*4    &  ~    \\      
Number of scans           & C3NIS    & I*4    &  ~    \\      
Number of scans done      & C3NSAMPLE & I*4    &  ~    \\      
Number of scan table 1 variables & C3NO\_SCAN\_VARS1 & I*4    & ~     \\      
Number of scan table 2 variables & C3NO\_SCAN\_VARS2 & I*4    & ~     \\      
Total time of scan        & C3SRT    & I*4    & SEC   \\      
Maximum number of map points done in a phase & C3MXP    & I*4    & ~     \\      
Maximum number of cycles in the scan & C3NCI    & I*4    &  ~    \\      
Number of cycles done in the scan    & C3NCYCLE & I*4    &  ~    \\      
Duration of each cycle               & C3CL     & I*4    & SEC   \\      
Total number of xy positions observed during a cycle & C3NCP    & I*4    & ~     \\      
Number of phase table variables      & C3NSV    & I*4    & ~     \\      
Number of phases per cycle           & C3PPC    & I*4    &  ~    \\      
No. of frontend output channels      & C3NFOC   & I*4    &  ~    \\      
Number of IF inputs to each section  & C3NOIFPBES  & I*4 &  ~    \\      
What's in the DATA array             & C3AOSOUTPUT & C*8 &  ~    \\      
BE input channels connected to this section        & C3BESCONN & I     &  ~    \\      
IF output channels connected to BE input channels  & C3BEINCON & I*4   &  ~    \\      
Number of channels per backend section             & C3LSPC    & I*4   &  ~    \\      
Subsystem nr to which each backend section belongs & C3BESSPEC & I*4   &  ~    \\      
Copy of frontend LO frequency per backend section  & C3BEFENULO & R*8  &  ~    \\      
Total IF per backend section                       & C3BETOTIF & R*8   &  ~    \\      
Copy of frontend sideband sign per backend section & C3BEFESB  & I*4   &  ~    \\      
Scan integration time                              & C3INTT   & I*4    &  ~    \\      
Number of backend output channels                  & C3NCH    & I*4    &  ~    \\      
Number of backend sections                         & C3NRS    & I*4    &  ~    \\      
Number of backend input channels                   & C3NRC    & I*4    &  ~    \\      
Backend configuration                              & C3CONFIGNR & I*4  &  ~    \\      
Description of output in DAS DATA                  & C3DASOUTPUT & C*8 &  ~    \\      
DAS calibration source for backend calibration     & C3DASCALSRC & C*6 &  ~    \\      
DAS calibration source for backend calibration     & C3DASSHFTFRAC &R*4&  ~    \\      
Subband overlap                                    & C3OVERLAP & R*4   &  ~    \\ \hline
\end{tabular}
 \\


                            
                            
\subsection{Class 4: Pointing Parameters}


\begin{tabular}{||l|l|l|l||} \hline
Item                      & Mnemonic & Format & Units \\ \hline
Character code of commanded centre                             & C4CSC    & C*16   &  ~    \\      
centre coords                                                  & C4CECO   & I*4    &  ~    \\      
Type of epoch, JULIAN, BESSELIAN or APPARENT                   & C4EPT    & C*16   &  ~    \\      
Centre moving flag (solar system object)                       & C4MCF    & L*1    &  ~    \\      
Date of the RA/DEC coordinates (1950)                          & C4EPH    & R*8    & YEAR  \\      
Right ascension of source for EPOCH                            & C4ERA    & R*8    & DEG   \\      
Declination of source for EPOCH                                & C4EDEC   & R*8    & DEG   \\      
Right Ascension of date                                        & C4RADATE & R*8    & DEG   \\      
Declination of date                                            & C4DECDATE& R*8    & DEG   \\      
Right ascension J2000                                          & C4RA2000 & R*8    & DEG   \\      
Declination J2000                                              & C4EDEC2000 & R*8  & DEG   \\      
Galactic longitude                                             & C4GL     & R*8    & DEG   \\      
Galactic latitude                                              & C4GB     & R*8    & DEG   \\      
Azimuth at observation date                                    & C4AZ     & R*8    & DEG   \\      
Elevation at observation date                                  & C4EL     & R*8    & DEG   \\      
Char. code for local x-y coord.system                          & C4LSC    & C*16   &  ~    \\      
Units of cell and mapping coordinates                          & C4ODCO   & C*16   &  ~    \\      
Angle between cell y axis and x-axis (CCW)                     & C4AXY    & R*8    & DEG   \\      
Commanded x centre position                                    & C4SX     & R*8    &  ~    \\      
Commanded y centre position                                    & C4SY     & R*8    &  ~    \\      
Reference x position                                           & C4RX     & R*8    &  ~    \\      
Reference y position                                           & C4RY     & R*8    &  ~    \\      
DAZ:Net Az offset at start                                     & C4AZERR  & R*8    &ARCSEC \\      
DEL:Net El offset at start                                     & C4ELERR  & R*8    &ARCSEC \\      
Secondary mirror is chopping                                   & C4SM     & L*1    &  ~    \\      
Secondary mirror chopping waveform                             & C4FUN    & C*10   &  ~    \\      
Secondary mirror chopping period                               & C4FRQ    & R*4    & Hz    \\      
Secondary mirror chopping coordinate system                    & C4SMCO   & C*2    &  ~    \\      
Secondary mirror chop throw                                    & C4THROW  & R*4    &ARCSEC \\      
Secondary mirror chop position angle                           & C4POSANG & R*4    & DEG   \\      
Secondary mirror offset parallel to lower axis (E-W Tilt)      & C4OFFS\_EW & R*4   &ARCSEC \\      
Secondary mirror offset parallel to upper axis (N-S Tilt)      & C4OFFS\_NS & R*4   &ARCSEC \\      
Secondary mirror absolute X position at observation start      & C4X      & R*4    & MM    \\      
Secondary mirror absolute Y position at observation start      & C4Y      & R*4    & MM    \\      
Secondary mirror absolute Z position at observation start      & C4Z      & R*4    & MM    \\      
Secondary mirror ew chop scale                                 & C4EW\_SCALE   & R*4 & ARCSEC/ENC\\      
Secondary mirror ns chop scale                                 & C4NS\_SCALE   & R*4 & ARCSEC/ENC\\      
Secondary mirror ew encoder value                              & C4EW\_ENCODER & I*4 & ENCODER   \\      
Secondary mirror ns encoder value                              & C4NS\_ENCODER & I*4 & ENCODER   \\      
Mounting of telescope                                          & C4MOCO        & C   &  ~        \\ \hline
\end{tabular}
 \\

 The items C4AXY, C4SX, C4SY, C4RX, C4RY, are stored as 
Real*8 but are actually Real*4 in the control task. \\

 Item C4CSC may also be source coordinate system.\\ 
 Item C4MOCO is defined as LOWER/UPPER axes, e.g; AZ/ALT.\\
 Item C4ODCO is offset definition code. \\
 Item C4CECO is centre coords where AZ=1; EQ=3; RD=4; RB=6; RJ=7; GA=8. \\

                                                                    
\subsection{Class 5: Environment Parameters}                           


\begin{tabular}{||l|l|l|l||} \hline
Item                                & Mnemonic & Format & Units \\ \hline
Ambient temperature                 & C5AT     & R*8    & DEG C \\
Mean atmospheric pressure           & C5PRS    & R*8    & MM HG \\
Mean atmospheric relative humidity  & C5RH     & R*8    & \%    \\ \hline
\end{tabular}
 \\

 The items C54AT, C5PRS, C5RH, are stored as Real*8 
but are actually Real*4 in the telescope task.




\subsection{Class 6: Mapping Parameters}


\begin{tabular}{||l|l|l|l||} \hline
Item                                          & Mnemonic & Format & Units \\ \hline
Local x-y AZ= 1;EQ=3;RD=4;RB=6;RJ=7;GA=8      & C6FC     & I*4    &  ~    \\      
Type of observation                           & C6ST     & C*16   &  ~    \\      
Map rows scanned in alternate directions?     & C6REV    & L*1    &  ~    \\      
Map rows are in X or Y direction              & C6SD     & C*16   &  ~    \\      
In first row x increases (T) or decreases (F) & C6XPOS   & L*1    &  ~    \\      
In first row y increases (T) or decreases (F) & C6YPOS   & L*1    &  ~    \\      
Number of sky points completed in the observation & C6NP & I*4    &  ~    \\      
Observation mode                              & C6MODE   & C*20   &  ~    \\      
Cycle reversal flag                           & C6CYCLREV& L*1    &  ~    \\      
Cell x dim,; descriptive origin item 1        & C6DX     & R*8    &  ~    \\      
Cell y dimension; descriptive origin item 2   & C6DY     & R*8    &  ~    \\      
Scanning angle, angle local vertical to x axis measured CW  & C6MSA    & R*8    &  ~    \\      
X map dimension; number of points in the x-direction        & C6XNP    & I*4    &  ~    \\      
Y map dimension; number of points in the y-direction        & C6YNP    & I*4    &  ~    \\      
X coordinate of the first map point                         & C6XGC    & R*4    &  ~    \\      
Y coordinate of the first map point                         & C6YGC    & R*4    &  ~    \\ \hline
\end{tabular}
 \\

 The items C6DX, C6DY, C6MSA are stored as Real*8 
but are actually Real*4 in the control task.
                                                    
                           


\subsection{Class 7: Velocity Parameters}


\begin{tabular}{||l|l|l|l||} \hline
Item                                        & Mnemonic  & Format & Units \\ \hline
Number of elements of vradial array         & C7SZVRAD  & I*4    & NO    \\      
Radial velocity of the source               & C7VR      & R*8    & KM/S  \\      
Bad channel value                           & C7BCV     & R*4    & IN    \\      
TEL's array for computing radial velocities & C7VRADIAL & R*8    &  ~    \\      
Filter                                      & C7FIL     & C*16   &  ~    \\      
Aperture                                    & C7AP      & C*16   &  ~    \\      
Sensitivity range of lockin                 & C7SNTVTYRG& C*16   &  ~    \\      
Lockin time constant                        & C7TIMECNST& C*16   &  ~    \\      
Lockin sensitivity in scale range units     & C7SNSTVTY & I*4    &  ~    \\      
Lockin phase                                & C7PHASE   & R*4    &  ~    \\      
CSO tau at 225GHz                           & C7TAU225  & R*4    &  ~    \\      
CSO tau time (YYMMDDHHMM)                   & C7TAUTIME & C      &  ~    \\      
Seeing at JCMT                              & C7SEEING  & R*4    &  ~    \\      
SAO seeing time (YYMMDDHHMM)                & C7SEETIME & C      &  ~    \\  \hline
\end{tabular}
 \\

                            
                            
\subsection{Class 8: Engineering Parameters}


\begin{tabular}{||l|l|l|l||} \hline
Item                                                 & Mnemonic & Format & Units \\ \hline
Ratio total power observed/incident on the telescope & C8AAE    & R*8    & \%    \\      
Fraction of beam in diffraction limited main beam    & C8ABE    & R*8    & \%    \\      
Antenna gain                                         & C8GN     & R*8    &  ~    \\      
Rear spillover and scattering efficiency             & C8EL     & R*8    &  ~    \\      
Forward spillover and scattering efficiency          & C8EF     & R*8    &  ~    \\  \hline
\end{tabular}
 \\



\subsection{Class 9: Data Parameters}

\begin{tabular}{||l|l|l|l||} \hline
Item                      & Mnemonic & Format & Units \\ \hline
   None                   &   ~      &   ~    &  ~    \\ \hline
\end{tabular}

                                                                                



\subsection{Class 10: Reciever Parameters}

\begin{tabular}{||l|l|l|l||} \hline
Item                      & Mnemonic & Format & Units \\ \hline
 None                     &    ~     &  ~     &  ~    \\ \hline
\end{tabular}

                                                                                

\subsection{Class 11: Phase Control Table}


\begin{tabular}{||l|l|l|l||} \hline
Item                                    & Mnemonic & Format & Units \\ \hline
Names of the cols. of phase table       & C11VD    & C*16   &  ~    \\      
Phase table: switching scheme dependent & C11PHA   & R*4    &  ~    \\ \hline
\end{tabular}
 \\




\subsection{Class 12: Phase Value Table}


\begin{tabular}{||l|l|l|l||} \hline
Item                                           & Mnemonic & Format & Units \\ \hline
Cold load temperature                          & C12TCOLD & R*4    & K     \\      
Ambient load temperature                       & C12TAMB  & R*4    & K     \\      
Velocity definition code - radio, optical, or relativistic & C12VDEF  & C*20   & ~     \\      
Velocity frame of reference - LSR, Bary-, Helio-, or Geo- centric   & C12VREF  & C*20  & ~     \\      
Units of spectrum data                         & C12CAL   & C*20   & ~     \\      
Calibration instrument used (FE, BE, or USER)  & C12CALTASK & C*20   & ~     \\      
Type of calibration (THREELOADS or TWOLOADS)   & C12CALTYPE & C*20   & ~     \\      
Way of calibrating the data (RATIO or DIFFERENCE) & C12REDMODE & C*20   & ~     \\      
Names of the cols. of scan table1              & C12SCAN\_VARS1 & C*16   & ~     \\      
Names of the cols. of scan table2              & C12SCAN\_VARS2 & C*16   & ~     \\      
Begin scan table                               & C12SCAN\_TABLE\_1 & R*4    &  ~    \\      
End scan table                                 & C12SCAN\_TABLE\_2 & R*4    &  ~    \\      
Centre frequencies (rest frame of source)      & C12CF      & R*8   & GHZ  \\      
Rest frequency                                 & C12RF      & R*8   & GHZ  \\      
BE input frequencies                           & C12INFREQ  & R*8   & GHZ  \\      
Frequency resolution                           & C12FR      & R*8    & MHZ  \\      
Bandwidth                                      & C12BW      & R*4    & MHZ  \\      
Receiver temperature                           & C12RT      & R*4    & K    \\      
System temperature                             & C12SST     & R*4    & K    \\      
Sky temperature at last calibration            & C12TSKY    & R*4    & K    \\      
Telescope temp. from last skydip               & C12TTEL    & R*4    & K    \\      
Gain value (kelvins per volt or equivalent)    & C12GAINS   & R*4    & K/V  \\      
Calibration temperature                        & C12CT      & R*4    & K    \\      
Water opacity                                  & C12WO      & R*4    & NEPER\\      
Sky transmission from last calibration         & C12ETASKY  & R*4    & ~    \\      
Ratio of signal sideband to image sideband sky transmission & C12ALPHA   & R*4    & ~    \\      
Normalizes signal sideband gain                & C12GS      & R*4    & ~   \\      
Telescope transmission                         & C12ETATEL  & R*4    & ~   \\      
Frontend-derived Tsky, image sideband          & C12TSKYIM  & R*4    & ~   \\      
Frontend-derived sky transmission              & C12ETASKYIM& R*4    & ~   \\      
Frontend-derived Tsys, image sideband          & C12TSYSIM  & R*4    & ~   \\      
Ratio of signal sideband to image sideband sky transmission & C12TASKY   & R*4    & ~   \\      
Correlation function mode                      & C12CM    & I*4    &   ~   \\      
Correlation bit mode                           & C12BM    & I*4    &   ~   \\ 
Centre frequencies (rest frame of source)      & C12CF    & R*8    & GHZ   \\      
Rest frequency; FITS: [Hz]; GSD: [GHz]         & C12RF    & R*8    & GHZ   \\      
Frequency resolution; FITS: [Hz]; GSD: [MHz]   & C12FR    & R*8    & MHZ   \\      
Bandwidth; FITS: [Hz]; GSD: [MHz]              & C12BW    & R*8    & MHZ   \\      
Calibration temperature                        & C12CT    & R*8    & K     \\      
Water opacity                                  & C12WO    & R*8    & NEPER \\ \hline
\end{tabular}
 \\

 The item C12CT as C1BKE.TCAL is saved as a Real*8 but us a Real*4 in the 
ukt14 task. 

                                                                      
                            
\subsection{Class 13: Phase Timing Table}


\begin{tabular}{||l|l|l|l||} \hline
Item                                                 & Mnemonic & Format & Units \\ \hline
Spectrum data                                        & C13DAT   & R*4    &  ~    \\      
Spectrum data                                        & C13DAT   & R*8    &  ~    \\      
Individual beam integrations                         & C13SPV   & R*8    &  ~   \\      
Raw error is accumulated over the scan               & C13RAW\_ERROR    & R*8    &  ~    \\      
Raw out of phase data samples in each phase          & C13SPV\_OP       & R*8    &  ~    \\      
Raw (out of phase) error, stored at end scan         & C13RAW\_ERROR\_OP & R*8    &  ~    \\      
array of responsivities                              & C13RESP  & R*4    &  ~    \\      
Phase data standard deviation                        & C13STD   & R*8    &  ~    \\      
Standard error                                       & C13ERR   & R*8    &  ~    \\ \hline
\end{tabular}
 \\

 The C13SPV is saved as Real*8 but is a real*4 in the DAS task, and
a Real*8 in the ukt14 task. 

 The C13DAT is listed with two types for saving, that for the DAS task
of Real*4 and that for the ukt14 task of Real*8.                                                                  
                                                              
        
\subsection{Class 14: Data Value Table}


\begin{tabular}{||l|l|l|l||} \hline
Item                             & Mnemonic & Format & Units \\ \hline
List of xy offsets for each scan & C14PHIST & R*4    & RUNS  \\  \hline
\end{tabular}
 \\




\subsection{Class 15: Pointing History Table}

\begin{tabular}{||l|l|l|l||} \hline
Item                      & Mnemonic & Format & Units \\ \hline
 None                     &    ~     &   ~    &   ~   \\ \hline
\end{tabular}
 \\


                                                                                


\subsection{Class 55: Inclinometry}


\begin{tabular}{||l|l|l|l||} \hline
Item                                               & Mnemonic & Format & Units \\ \hline
DAS number of phases for interferometry observing  & C55NPH   & I*4    &  ~    \\      
DAS number of phases for interferometry observing  & C55NCYC  & I*4    &  ~    \\      
DAS number of phases for interferometry observing  & C55NINT  & I*4    &  ~    \\      
DAS number of correlation cycles                   & C55NCORR & I*4    &  ~    \\      
RXJ X length of projected baseline in metres       & C55LX     & R*8    &  ~    \\      
RXJ Y length of projected baseline in metres       & C55LY     & R*8    &  ~    \\      
RXJ Z length of projected baseline in metres       & C55LZ     & R*8    &  ~    \\      
RXJ Coeff of sin term in exp for fringe rate       & C55A      & R*8    &  M    \\      
RXJ Coeff of cos term in exp for fringe rate       & C55B      & R*8    &  M    \\      
RXJ Coeff of constant term in exp for fringe rate  & C55C      & R*8    &  M    \\      
RXJ Const term to compensate for differing telescope focal lengths  & C55D      & R*8    &  M    \\      
RXJ Delay setting of RXJ micro for CSO side        & C55CSOSW  & I*4    &  ~    \\      
RXJ Delay setting of RXJ micro for JCMT side       & C55JCMTSW & I*4    &  ~    \\      
RXJ Number of the tick on which integration started& C55SECOND & I*4    &  ~    \\      
CSO Position of absorber IN or OUT                 & C55ABSORB & C*1    &  ~    \\      
CSO TAU value                                      & C55TAU    & R*4    &  ~    \\      
CSO Position offset in az                          & C55DAZ    & R*4    &ARCSEC \\      
CSO Position offset in elevation                   & C55DEL    & R*4    &ARCSEC \\      
CSO RA                                             & C55RA     & R*8    &  ~    \\      
CSO Dec                                            & C55DEC    & R*8    &  ~    \\      
CSO Epoch of CSO RA and Dec                        & C55EPOCH  & R*8    &  ~    \\      
CSO Pointing offset in az                          & C55PAZ    & R*4    &ARCSEC \\      
CSO Pointing offset in elevation                   & C55PEL    & R*4    &ARCSEC \\      
CSO Track mode of telescope Y or N                 & C55TRACK  & C*1    &  ~    \\      
CSO Focus mode of CSO                              & C55FMODE  & C*1    &  ~    \\      
CSO X position of focus                            & C55FX     & R*4    &  ~    \\      
CSO Y position of focus                            & C55FY     & R*4    &  ~    \\      
CSO Z position of focus                            & C55FZ     & R*4    &  ~    \\      
CSO LSR velocity of source                         & C55VLSR   & R*4    & KM/S  \\      
CSO  velocity offset                               & C55VOFF   & R*4    & KM/S  \\      
CSO radial velocity                                & C55VRAD   & R*4    & KM/S  \\      
CSO Phase lock status L or U                       & C55PLOCK  & C*1    &  ~    \\      
CSO Rest frequency of line                         & C55RFREQ  & R*8    & GHz   \\      
CSO IF frequency                                   & C55IFFREQ   & R*8    & GHz   \\      
CSO LO frequency                                   & C55LOFREQ   & R*8    & GHz   \\      
CSO frequency offset                               & C55FREQOFF  & R*8    & GHz   \\      
CSO Sideband U or L                                & C55SIDEBAND & C*1    &  ~    \\      
CSO Multiplier Harmonic number                     & C55MHN      & I*4    &  ~    \\      
CSO overall status 0 = bad  1 = good               & C55CSOSTATUS& I*4    &  ~    \\      
CENTRE\_AZ from tel sdis array                     & C55TELAZ    & R*8    &  ~    \\      
CENTRE\_EL from tel sdis array                     & C55TELEL    & R*8    &  ~    \\      
Observing frequencies                              & C55FENUOBS  & R*8    &  ~    \\      
FE side band signs                                 & C55FESBSIGN & I*4    &  ~    \\      
FE LO frequency                                    & C55FENULO   & R*8    &  ~    \\      
DAS total power measurement on hot load            & C55HOTPOWER & R*4    &  ~    \\      
DAS data processing done                           & C55DASPRBIT & I*4    &  ~    \\      
Description of where processing is done            & C55DASPRLOC & C*10   &  ~    \\      
DAS total power measurement on hot load            & C55SKYPOWER & R*4    &  ~    \\      
Samples to store for cross correlation mode        & C55SAM      & R*4    &  ~    \\      
DAS total power per subband per integration        & C55POWER    & R*4    &  ~    \\ \hline
\end{tabular}
                                                                      

\subsection{No Class}

                    
\begin{tabular}{||l|l|l|l||} \hline
Item                                & Mnemonic  & Format & Units \\ \hline
Position angle of cell y axis (CCW) & CELL\_V2Y & R*8    & DEG   \\ \hline
\end{tabular}
 \\
        

\section{AOB}


\end{document}
